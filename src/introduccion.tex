\section{Introducci\'on}
Cuando una persona es propietaria de un vehiculo es com\'un realizar mantenimiento
peri\'odico al mismo a lo largo de su vida \'util, ya sea por motivos preventivos,
legales o por un accidente. Este tipo de mantenimiento entre otros son conocidos como
asistencias vehiculares, las cuales son realizadas diariamente en los diferentes
concesionarios autorizados.

La asistencia vehicular esta compuesta por una gran variedad de servicios, los cuales
tienen su correspondiente costo, duraci\'on y eventual descuento.

El proceso de la asistencia vehicular es iniciada a trav\'es de la solicitud del propietario,
donde se especif\'ica la fecha y cantidad de servicios solicitados, en la mayoria de los
casos la asistencia es completada exitosamente, pero en algunos otros no es completada.

El presente trabajo final de materia pretende determinar que factores influyen para que
una asistencia vehicular no sea completada mediante la construcci\'on de un \'arbol de
decisi\'on utilizando la herramienta Rapidminer.

